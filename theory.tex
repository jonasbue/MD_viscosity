\section{Theory}
\label{sec:theory}
\subsection{Theoretical description of hard-sphere fluids}
Theroretical models of hard-sphere fluids are important descriptions of fluids.
% Write a proper introduction.
Of particular interest for this project is the Enskog equation for the viscosity of fluids, and its generalization to hard-sphere fluid mixtures.

\subsection{Simulation hard-sphere fluids}
Several methods exist for simulating fluids consisting of hard spheres.
While Monte Carlo methods are relatively simple to use with hard sphere potentials -- that is, spherical particles (molecules) wia non-continous interaction potential of the form
\begin{equation}
    u = 
    \begin{cases}
        \infty, & r < \sigma \\
        0, & r > \sigma,
    \end{cases}
\end{equation}
molecular dynamics methods allow calculating dynamical and out-of-equilibrium properties of a system \cite{allen:MD_sim}.

Furthermore, Jover et. al \cite{jover:pseudo_hard} has shown that a pseudo hard Mie (or generalized Lennard-Jones) potential
\begin{equation}
    u_{\text{Mie}}(r) = 
        \frac{\lambda_r}{\lambda_r - \lambda_a}
        \left(\frac{\lambda_r}{\lambda_a}\right)
        ^{\frac{\lambda_a}{\lambda_r - \lambda_a}}
        \epsilon \left[
            \left(\frac{\sigma}{r}\right)^{\lambda_r} -
            \left(\frac{\sigma}{r}\right)^{\lambda_a}
        \right],
\end{equation}
can approximate a hard-sphere interaction potential well, when the repulsive part of the Mie potential is isolated, by shifting it upwards by its minimal value (the well depth $\epsilon$), and cutting it off here, setting the potential to zero once it has reached its minimum.
This gives a steep non-negative potential of the form
\begin{equation}
    u_{(\lambda_a, \lambda_b)}(r) = 
    \begin{cases}
        \frac{\lambda_r}{\lambda_r - \lambda_a}
        \left(\frac{\lambda_r}{\lambda_a}\right)
        ^{\frac{\lambda_a}{\lambda_r - \lambda_a}}
        \epsilon \left[
            \left(\frac{\sigma}{r}\right)^{\lambda_r} -
            \left(\frac{\sigma}{r}\right)^{\lambda_a}
        \right]
        + \epsilon,
            & r < \sigma \left(
                \frac{\lambda_r}{\lambda_a}
            \right)^\frac{1}{\lambda_r - \lambda_a} \\
        0,  & r > \sigma \left(
                \frac{\lambda_r}{\lambda_a}
            \right)^\frac{1}{\lambda_r - \lambda_a},
    \end{cases}
\end{equation}
closely resembling that of an infinitely steep hard wall potential.
Jover et. al chose the exponents $(\lambda_r, \lambda_a) = (50, 49)$ as a compromise between the faithfullness of the pseudo hard representation and comptational speed:
Higher exponents will give steeper potentials, but at the expense of making simulations more time-consuming, because shorter time steps are needed.
For clarity, the Mie (50, 49) potential has the form
\begin{equation}
    u_{(50, 49)}(r) = 
    \begin{cases}
        50
        \left(\frac{50}{49}\right)
        ^{49}
        \epsilon \left[
            \left(\frac{\sigma}{r}\right)^{50} -
            \left(\frac{\sigma}{r}\right)^{49}
        \right]
        + \epsilon,
            & r < \frac{50}{49} \sigma\\
        0,  & r > \frac{50}{49} \sigma.
    \end{cases}
\end{equation}

Pousaneh and de Wijn \cite{pousaneh:shear_viscosity} have shown that such a pseudo-hard wall potential can be used to model viscosity for a one-component, and that the obtained viscosity is in agreement with Enskog theory.
