\chapter{Theory}
\label{sec:theory}

\section{Kinetic theory}
% TODO: Skriv en litt generell intro.
% Boltzmannlikningen
% Antakelser
% Enskogteori
% Likevekt

\section{Hard-sphere fluids}
Hard-sphere (HS) fluids are fluids consisting of spherical particles 
interacting via the non-continuous potential 
\begin{equation}
    \label{eq:hard_sphere_potential}
    u = 
    \begin{cases}
        \infty, & r < \sigma \\
        0, & r > \sigma.
    \end{cases}
\end{equation}
$u$ is the interaction potential between two particles, 
$r$ is the distance between them, 
and $\sigma$ is the diameter of the particles.
This potential makes it possible to analytically describe 
certain fluid properties using statistical mechanics. 
The HS potential is especially useful when modeling 
transport properties such as viscosity.


\section{Transport properties}
% TODO: Fyll inn tekst her.
% Nevn noe av Enskog-informasjonen fra neste seksjon her.
% Generelt om transportegenskaper, enskogteori som en metode for å finne dem
% Kanskje nevne andre metoder for å regne ut transportegenskaper?
% Viskositet som transportegenskap


\section{The viscosity of hard-sphere fluids}
The central equation modeling the viscosity of HS fluids is the Enskog equation. 
This equation applies to one-component HS fluids consisting only of particles 
with the same mass $m$ and radius $\sigma$.
The viscosity of such fluids is \cite{ref:pippo:composition_dependence}
\begin{equation}
    \label{eq:one_component_viscosity}
    \eta(\rho,T) 
        = \eta_0 \left[g^{-1}(\sigma) 
        + 0.8   \, V_{\text{excl}}      \,\rho 
        + 0.776 \, V_{\text{excl}}^2    \,\rho^2 g(\sigma) 
        \right].
\end{equation}
%\begin{equation}
%    \begin{split}
%        \eta(n,T)
%            &= \frac{\eta^0}{\chi} (1+\frac{1}{2} \alpha \chi n)^2 + \frac{3}{5} \bar{\omega}, \quad \text{where}\\
%        \alpha
%            &= \frac{8}{15} \pi \sigma^3, \quad \text{and} \\
%        \bar{\omega}
%            &= \frac{4}{9} n^2 \sigma^4 \chi \sqrt{\pi m \kB T}.
%    \end{split}
%\end{equation}
Here, 
\begin{equation}
    \eta_0 = \eta(0, T) = 
    \frac{5}{16\sigma^2} \sqrt{\frac{m \kB T}{\pi}}
\end{equation}
is the viscosity of the fluid in the zero-density limit.
$V_{\text{excl}}$, the excluded volume, 
is the part of the fluid which one cannot occupy,
because it is occuied by other particles.
$V_{\text{excl}}$ is commonly assumed to be 
the volume of all particles in the fluid.
This however, is only correct in the zero-density limit. 
When the fluid density is higher, 
the space in between particles may also be unavailable.
% TODO: Make a figure.

Lastly, in equation \eqref{eq:one_component_viscosity}, 
$g(\sigma)$ is a radial distribution function at contact.
$g(\sigma)$ is the probability distribution of particles around one particle in the fluid.
This determines how the collision frequency depends on the density of the fluid.
$g(\sigma)$ can be found for example using the system's equation of state, 
as done in \cite{ref:pippo:composition_dependence}.
This method is referred to as Modified Enskog Theory.

% TODO:
% Flytt biten om Enskog til: The viscosity of hard sphere fluids. 
% Deler av dette kan også nevnes i seksjonen om transportegenskaper. 
% Det er et generelt problem det du beskriver om MFT (mean free time).

In Enskog theory, a central assumption is that there is no correlation between different collisions. 
This is known as molecular chaos.
This means that the mean free time between collisions is much larger than the collision duration. 
The assumption of molecular chaos breaks down at high fluid densities, when collisions are too frequent to be uncorrelated.
Furthermore, it does not work for real fluids with long range continuous interaction potentials.
However, the assumption is more acceptable for low density HS-fluids. 


\section{Viscosity of hard-sphere fluid mixtures}
The Enskog equation was generalized by Thorne to describe 
two-component fluid mixtures\cite{ref:chapman:non_uniform_gases}.
A further generalization to mixtures of arbitrary component numbers 
has been performed by Tham and Gubbins\cite{ref:tham:fluid_mixtures}.
The results are outlined below, 
as presented in \cite{ref:pippo:composition_dependence}.

The viscosity of a dense binary mixture of two hard-sphere fluids is given by
\begin{equation}
    \eta_{\text{mix}} 
        = \left(
            \frac{y_1^2}{H_{11}} 
            + \frac{y_2^2}{H_{22}} 
            - \frac{2 y_1 y_2 H_{12}}{H_{11} H_{22}}
        \right)
        \left(
            1 - \frac{H_{12}^2}{H_{11} H_{22}}
        \right)^{-1}
        + \frac{3}{5} \bar{\omega}_{\text{mix}},
\end{equation}
where
\begin{equation}
    \label{eq:viscosity_binary}
    y_1 
        = x_1 \left(
            1   + \frac{1}{2} x_1 \alpha_{11} \chi_{11} n 
                + \frac{m_2}{m_1 + m_2} x_2 \alpha_{12} \chi_{12} n
        \right), 
\end{equation}
and
\begin{equation}
    \begin{split}
        H_{12} &= H_{21}
                =   -\frac{2 x_1 x_2 \chi_{12}}{\eta^0_{12}}
                    \cdot \frac{m_1 m_2}{(m_1 + m_2)^2}
                    \left( \frac{5}{3A^*_{12}} - 1 \right), \\
        H_{11}
                &=  -\frac{x_1^2 \chi_{11}}{\eta^0_1}
                    +\frac{2 x_1 x_2 \chi_{12}}{\eta^0_{12}}
                    \cdot \frac{m_1 m_2}{(m_1 + m_2)^2}
                    \left( \frac{5}{3A^*_{12}} + \frac{m_2}{m_1} \right). \\
    \end{split}
\end{equation}
$y_2$ and $H_{22}$ follow from exchanging the subscripts in $y_1$ and $H_{11}$ 
respectively.
$A^*_{12}$ is a dimensionless ratio of collision integrals (of type ${ij}$).
For hard spheres, $A^*_{12}$ is exactly unity, 
and for other forms of interaction, it is close to unity.

$\chi_{ij} = \chi_{ji}$ are radial distrubution functions for molecules 
of type $i$ colliding with molecules of type $j$, 
similar to the one-component function $\chi$ above.
Finally, $\bar{\omega}_{\text{mix}}$ can be written
\begin{equation}
    \begin{split}
        \bar{\omega}_{\text{mix}} 
            &= x_1^2 \bar{\omega}_{11} 
            + x_1 x_2 \bar{\omega}_{12} 
            + x_2^2 \bar{\omega}_{22}, \text{where}\\
        \bar{\omega}_{ij} 
            &= \frac{4}{9} n^2 \sigma_{ij}^4 \chi_{ij} 
            \sqrt{\frac{2\pi m_1 m_2 \kB T}{m_1 + m_2}} 
            \text{for} i,j = 1,2.
    \end{split}
\end{equation}

% TODO: Ellaborate on this, after discussing with Roberto.
%       In particular, say something about the physical meaning of every
%       "russian doll" in the equation.

% TODO: Should I include the multi-component equation as well?
%       Or maybe just include one and many components, and skip two-compnent?
%       ANSWER: Start with multi-compnent, then reduce to two.
%       COMMENT TO ANSWER: Since the two-component version is more important
%       to this project, and since it was derived first, I think it's more
%       natural to present the two-component equation first, like Pippo does.

\section{Simulation of hard-sphere fluids}
Several methods exist for simulating fluids consisting of rigid spheres.
They can be divided into two categories, namely Monte Carlo (MC) models,
and molecular dynamics (MD) models.
MC methods sample statistical ensembles using random walk,  % TODO: Find a better term than random walk.
while MD methods use Newton's equations of motion to compute the deterministic
path of a large collection of particles \cite{ref:allen:MD_sim}.

Monte Carlo methods are typically easier to use with hard-sphere potentials,
but can not be used to compute transport coefficients. 
% Can't or can with difficulty? 
% That's an important distinction.
Molecular dynamics methods on the other hand, allow calculating dynamical 
and out-of-equilibrium properties of a system \cite{ref:allen:MD_sim}.
To compute the viscosity of a system, MD methods should therefore be used.
This section gives an introduction to both Monte Carlo and molecular dynamics,
focusing especially on molecular dynamics for hard and pseudo-hard spheres.

% Mer generelt om MD (inkl hendelsesdrevent)
\section{Event-driven molecular dynamics simulation}
A simple method for hard sphere simulations in molecular dynamics is an
event-driven HS simulation.
Key elements of this simultion is outlined below.

Compute the time until a collision occurs, for all particles in the system.
All collisions are stored in a list containing at least information about
the time of the collision, and the identity of the two involved particles.
Then, the time until the earliest collision is identified by searching through the list.
Using Newton's equations of motion, all atoms are propagated freely until
the collision happens.
Through conservation laws, the colliding particles' velocities are then updated.
Their next collisions are then added to the list of upcoming collisions.
This process is repeated, and all atom positions are updated with the time 
until the next collision in the list.

It should be noted that after a collision happens, other collisions involving
either of the two particles will be invalid.
These should be discarded from the list.

\section{Continuous potential MD simulation methods}
Several powerful and efficient molecular dynamics programmes are easily 
available, including LAMMPS, GROMACS, DL\_POLY and NAND. % TODO add references.
These do, however, not handle discontinous potentials.
Thus, the event-driven method is not supported by any of these programmes \cite{ref:allen:MD_sim}.
In order to utilize the efficiency of the available MD software, it is more convenient to use a different method.
The hard sphere potential can be approximated with a steep but continuous interaction potential instead.
Then, the particle positions are updated using numerical integration methods with short, finite time steps.

Additionally, once there are long-range interaction forces 
between particles, then interactions occur at all times.
In this case, the event driven simulation does not 
work, and integration methods are required.
Therefore, continuous potential modelling 
is much more useful for realistic fluid models.

Several potentials are possible to use as hard-sphere approximations.
The Lennard-Jones potential 
\begin{equation}
    \label{eq:lennard_jones_potential}
    u_{\text{LJ}}(r) = 
        4 \epsilon \left[
            \left(\frac{\sigma}{r}\right)^{12} -
            \left(\frac{\sigma}{r}\right)^{6}
        \right]
\end{equation}
is a well-known example of historical importance.
Here, $\epsilon$ is the well depth -- the minimal value of the potential.
As in \eqref{eq:hard_sphere_potential}, 
$\sigma$ and $r$ are the diameter and relative distance of the interacting particles.

If the attractive part, 
$\left[- \left(\frac{\sigma}{r}\right)^{6}\right]$
in equation \eqref{eq:lennard_jones_potential} is removed from the potential, 
then $u_{\text{LJ}}(r)$ 
can represent purely repulsive interactions.
This is done by shifting the potential
%The repulsive part of the Mie potential can be isolated by shifting it 
upwards by its minimal value $\epsilon$ and cutting it off there. 
Thus, the potential is excactly zero once it has reached its minimum.
The resulting potential
\begin{equation}
    \label{eq:WCA_potential}
    u_{WCA}(r) = 
    \begin{cases}
        4 \epsilon \left[
            \left(\frac{\sigma}{r}\right)^{12} -
            \left(\frac{\sigma}{r}\right)^{6}
        \right]
        + \epsilon,
            & r < 2^\frac{1}{6} \sigma\\
        0,  & r > 2^\frac{1}{6}\sigma.
    \end{cases}
\end{equation}
is known as a WCA-potential.
This serves as an approximation to a hard sphere potential.
% This is explained below. 
% TODO: Move the explanation here, so that everything is chronological.

Increased computer efficiency makes it more appropriate to use 
steeper potentials than the WCA-potential.
In particular, Jover et al. \cite{ref:jover:pseudo_hard} 
has proposed the Mie (or generalized Lennard-Jones) potential
\begin{equation}
    u_{\text{Mie}}(r) = 
        \frac{\lambda_r}{\lambda_r - \lambda_a}
        \left(\frac{\lambda_r}{\lambda_a}\right)
        ^{\frac{\lambda_a}{\lambda_r - \lambda_a}}
        \epsilon \left[
            \left(\frac{\sigma}{r}\right)^{\lambda_r} -
            \left(\frac{\sigma}{r}\right)^{\lambda_a}
        \right],
\end{equation}
as an approximation to the hard-sphere interaction potential.
%$\epsilon$ is the depth of the potential at it's minimal value, 
% What's a better word than "strength"??
The exponents $\lambda_r$ and $\lambda_a$ define the strength of the 
repulsive and attractive parts of the potential.

Cutting and shifting the $u_{\text{Mie}}$ potential as done in equations
\eqref{eq:lennard_jones_potential} and \eqref{eq:WCA_potential},
gives a steep non-negative potential of the form
\begin{equation}
    u_{(\lambda_a, \lambda_b)}(r) = 
    \begin{cases}
        \frac{\lambda_r}{\lambda_r - \lambda_a}
        \left(\frac{\lambda_r}{\lambda_a}\right)
        ^{\frac{\lambda_a}{\lambda_r - \lambda_a}}
        \epsilon \left[
            \left(\frac{\sigma}{r}\right)^{\lambda_r} -
            \left(\frac{\sigma}{r}\right)^{\lambda_a}
        \right]
        + \epsilon,
            & r < \sigma \left(
                \frac{\lambda_r}{\lambda_a}
            \right)^\frac{1}{\lambda_r - \lambda_a} \\
        0,  & r > \sigma \left(
                \frac{\lambda_r}{\lambda_a}
            \right)^\frac{1}{\lambda_r - \lambda_a},
    \end{cases}
\end{equation}
closely resembling that of an infinitely steep hard wall potential (equation \eqref{eq:hard_sphere_potential}).
This potential is referred to as a pseudo hard-sphere (PHS) potential.

Jover et al. chose the exponents $(\lambda_r, \lambda_a) = (50, 49)$, 
as a compromise between faithfulness of the pseudo hard representation 
towards the perfectly hard wall, and computational speed.
Higher exponents will produce a steeper repulsion. 
This however, comes at a cost. 
The steeper the potential, the shorter time steps are 
needed to ensure that the computations are precice.
% Definer hva presist betyr.
Therefore, steeper repulsions are computationally more expensive to simulate.

Writing it out for clarity, the Mie (50, 49)-potential has the form
\begin{equation}
    u_{(50, 49)}(r) = 
    \begin{cases}
        50
        \left(\frac{50}{49}\right)
        ^{49}
        \epsilon \left[
            \left(\frac{\sigma}{r}\right)^{50} -
            \left(\frac{\sigma}{r}\right)^{49}
        \right]
        + \epsilon,
            & r < \frac{50}{49} \sigma\\
        0,  & r > \frac{50}{49} \sigma.
    \end{cases}
\end{equation}

Pousaneh and de Wijn \cite{ref:pousaneh:shear_viscosity} 
have shown that such a pseudo-hard sphere potential 
can be used to model viscosity for a one-component hard-sphere fluid, 
and that the obtained viscosity is in agreement with Enskog theory.
% However, to the best of our knowledge, such a confirmation has not 
% been published for fluids of more than one component. 
% C: Detter er ikke sjekket, men er jo delvis det du skal gjøre.

\section{Measuring the viscosity of a fluid in NEMD}
Müller-Plathe \cite{ref:mullerplathe:reversing_the_perturbation} 
has proposed a method of computing the viscosity of a fluid in 
nonequilibrium molecular dynamics (NEMD) simulations.
% TODO: Describe method, including figures.

Consider a rectangular box with sides of length $(L_x, L_y, L_z)$ 
with periodic boundary conditions.
The box contains a hard sphere fluid at equillibrium.
We divide the system into two equal slabs, their border 
being parallell to the $xy$-plane at height $z = L_z/2$.
Now, we can define three parallell planes, at 
$z = \{0, L_z/2, L_z\}$, which are the edges of the two slabs.
At these edges, we will control the velocity of the particles, 
making sure that the particles flow in the directions
\begin{equation}
    \label{eq:velocity_directions}
    \hat{u}_x(z) =
    \begin{cases}
        +\hat{x}, \,z = L_z,    \\
        -\hat{x}, \,z = L_z/2,  \\
        +\hat{x}, \,z = 0,    \\
    \end{cases}
\end{equation}
as shown in figure (??). %TODO: Figure.
Here, $u_x(z)$ denotes the average velocity component 
in $x$-direction, of the particles at height $z$.
The hat ``$\,\, \hat{ } \,\,$'' denotes 
directional vectors of length unity.

The particle flows are adjusted through an unphysical process known as a 
\textbf{reverse perturbation}\cite{ref:mullerplathe:reversing_the_perturbation},
as follows:

Find the particle in the $z = 0$ slab edge with the lowest 
(most negative) velocity component in the $+x$-direction.
Correspondingly, find the particle in the $z = L/2$ border 
with the largest velocity component in the $+x$-direction.
Then, swap the momenta of these two particles.
Repeat the process for the top slab, swapping the smallest momentum 
in the $z = L$ edge with the largest momentum in the $z = L/2$ edge.
This process is then repeated periodically.
The repeated reverse perturbations cause a velocity profile in the fluid,
as shown in figure (??). % TODO: Figure.
At the edges of the slabs, the velocity directions 
is given by equation \eqref{eq:velocity_directions}.

The velocity profile will cause a momentum flux in $z$-direction,
\begin{equation}
    \label{eq:momentum_flux}
    j_z = -\eta \frac{\partial u_x}{\partial z}.
\end{equation}
This flux is proportional to the viscosity of the fluid, and is measurable.
Thus, the viscosity can be calculated from velocity data generated in the 
MD simulation.

