\section{Theory}
\label{sec:theory}
\subsection{Theoretical description of hard-sphere fluids}
% TODO: Write a proper introduction.
%
Enskog theory describes transport properties of hard sphere fluids, as an approximation to real gas behaviour.
Importantly, Enskog theory assumes that there is no correlation between different collisions, 
meaning that the mean free time between collisions is much larger than the collision duration.
While this assumption gives useful results, it also means that Enskog theory breaks down at high fluid densities.
% TODO: Write more and add a source.

Of particular interest for this work is the Enskog equation for the viscosity of such fluids,
consisting of one single type of particle (one fluid component), of mass $m$ and diameter $\sigma$.
\begin{equation}
    \begin{split}
        \eta(n,T) 
            &= \frac{\eta^0}{\chi} (1+\frac{1}{2} \alpha \chi n)^2 + \frac{3}{5} \bar{\omega}, \quad \text{where}\\
        \alpha 
            &= \frac{8}{15} \pi \sigma^3, \quad \text{and} \\
        \bar{\omega} 
            &= \frac{4}{9} n^2 \sigma^4 \chi \sqrt{\pi m \kB T}.
    \end{split}
\end{equation}
Here, $\eta^0 = \eta(0, T)$ is the viscosity in the zero-density limit, 
and $\chi$ is a radial distribution function at contact, 
describing the statistical distribution of particles around a pair of colliding particles.
This determines how the collision frequency depends on the density of the fluid.

The Enskog equation was generalized by Thorne to describe two-component 
fluid mixtures\cite{ref:chapman:non_uniform_gases}, and a further 
generalization to mixtures of arbitrary component numbers has been 
performed by Tham and Gubbins\cite{ref:tham:fluid_mixtures}.
The results are outlined below, as presented in \cite{ref:pippo:composition_dependence}.

The viscosity of a dense binary mixture of hard-sphere fluids is given by
\begin{equation}
    \eta_{\text{mix}} 
        = \left(
            \frac{y_1^2}{H_{11}} + \frac{y_2^2}{H_{22}} - \frac{2 y_1 y_2 H_{12}}{H_{11} H_{22}}
        \right)
        \left(
            1 - \frac{H_{12}^2}{H_{11} H_{22}}
        \right)^{-1}
        + \frac{3}{5} \bar{\omega}_{\text{mix}},
\end{equation}
where
\begin{equation}
    \label{eq:viscosity_binary}
    y_1 
        = x_1 \left(
            1 + \frac{1}{2} x_1 \alpha_{11} \chi_{11} n + \frac{m_2}{m_1 + m_2} x_2 \alpha_{12} \chi_{12} n
        \right), 
\end{equation}
and
\begin{equation}
    \begin{split}
        H_{12} &= H_{21}
                =   -\frac{2 x_1 x_2 \chi_{12}}{\eta^0_{12}}
                    \cdot \frac{m_1 m_2}{(m_1 + m_2)^2}
                    \left( \frac{5}{3A^*_{12}} - 1 \right), \\
        H_{11}
                &=  -\frac{x_1^2 \chi_{11}}{\eta^0_1}
                    +\frac{2 x_1 x_2 \chi_{12}}{\eta^0_{12}}
                    \cdot \frac{m_1 m_2}{(m_1 + m_2)^2}
                    \left( \frac{5}{3A^*_{12}} + \frac{m_2}{m_1} \right), \\
    \end{split}
\end{equation}
and where $y_2$ and $H_{22}$ follows from exchanging the subscripts in $y_1$ and $H_{11}$ respectively.
$A^*_{12}$ is a dimensionless ratio of collision integrals (of type ${ij}$).
% TODO: See if more ellaboration on this exists.
For hard spheres, $A^*_{12}$ is exactly unity, and for other forms of interaction, it is close to unity.
$\chi_{ij} = \chi_{ji}$ are radial distrubution functions for molecules of type $i$ colliding with molecules of type $j$.
Finally, $\bar{\omega}_{\text{mix}}$ can be written
\begin{equation}
    \begin{split}
        \bar{\omega}_{\text{mix}} 
            &= x_1^2 \bar{\omega}_{11} + x_1 x_2 \bar{\omega}_{12} + x_2^2 \bar{\omega}_{22}, \text{where}\\
        \bar{\omega}_{ij} 
            &= \frac{4}{9} n^2 \sigma_{ij}^4 \chi_{ij} \sqrt{\frac{2\pi m_1 m_2 \kB T}{m_1 + m_2}} \text{for} i,j = 1,2.
    \end{split}
\end{equation}

%Equation \ref{eq:viscosity_binary} has also been generalized to any number of components by \cite{ref:tham:fluid_mixtures}.
% TODO: Should I include this as well?
%       Or maybe just include one and multiple components?

\subsection{Simulation of hard-sphere fluids}
Several methods exist for simulating fluids consisting of hard spheres.
While Monte Carlo methods are relatively simple to use with hard sphere potentials, 
that is, spherical particles (molecules) interacting via the non-continuous potential 
\begin{equation}
    u = 
    \begin{cases}
        \infty, & r < \sigma \\
        0, & r > \sigma,
    \end{cases}
\end{equation}
molecular dynamics methods allow calculating dynamical and out-of-equilibrium properties of a system \cite{ref:allen:MD_sim}.

Furthermore, Jover et al. \cite{ref:jover:pseudo_hard} has shown that a pseudo hard Mie (or generalized Lennard-Jones) potential
\begin{equation}
    u_{\text{Mie}}(r) = 
        \frac{\lambda_r}{\lambda_r - \lambda_a}
        \left(\frac{\lambda_r}{\lambda_a}\right)
        ^{\frac{\lambda_a}{\lambda_r - \lambda_a}}
        \epsilon \left[
            \left(\frac{\sigma}{r}\right)^{\lambda_r} -
            \left(\frac{\sigma}{r}\right)^{\lambda_a}
        \right],
\end{equation}
can approximate a hard-sphere interaction potential well, when the repulsive part of the Mie potential is isolated, by shifting it upwards by its minimal value (the well depth $\epsilon$), and cutting it off here, setting the potential to zero once it has reached its minimum.
This gives a steep non-negative potential of the form
\begin{equation}
    u_{(\lambda_a, \lambda_b)}(r) = 
    \begin{cases}
        \frac{\lambda_r}{\lambda_r - \lambda_a}
        \left(\frac{\lambda_r}{\lambda_a}\right)
        ^{\frac{\lambda_a}{\lambda_r - \lambda_a}}
        \epsilon \left[
            \left(\frac{\sigma}{r}\right)^{\lambda_r} -
            \left(\frac{\sigma}{r}\right)^{\lambda_a}
        \right]
        + \epsilon,
            & r < \sigma \left(
                \frac{\lambda_r}{\lambda_a}
            \right)^\frac{1}{\lambda_r - \lambda_a} \\
        0,  & r > \sigma \left(
                \frac{\lambda_r}{\lambda_a}
            \right)^\frac{1}{\lambda_r - \lambda_a},
    \end{cases}
\end{equation}
closely resembling that of an infinitely steep hard wall potential.
Jover et al. chose the exponents $(\lambda_r, \lambda_a) = (50, 49)$ as a compromise between the faithfulness of the pseudo hard representation and computational speed:
Higher exponents will give steeper potentials, but at the expense of making simulations more time-consuming, because shorter time steps are needed.
For clarity, the Mie (50, 49) potential has the form
\begin{equation}
    u_{(50, 49)}(r) = 
    \begin{cases}
        50
        \left(\frac{50}{49}\right)
        ^{49}
        \epsilon \left[
            \left(\frac{\sigma}{r}\right)^{50} -
            \left(\frac{\sigma}{r}\right)^{49}
        \right]
        + \epsilon,
            & r < \frac{50}{49} \sigma\\
        0,  & r > \frac{50}{49} \sigma.
    \end{cases}
\end{equation}

Pousaneh and de Wijn \cite{ref:pousaneh:shear_viscosity} have shown that such a pseudo-hard wall potential can be used to model viscosity for a one-component, and that the obtained viscosity is in agreement with Enskog theory.
