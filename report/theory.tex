\chapter{Theoretical descriptions of hard-sphere fluids}
\label{sec:theory}

\section{Transport theory}
\label{sec:theory:transport_theory}
Physical systems that are out of equilibrium
can be described by the transport of certain
quantities in the phase space. These quantities
are known as \emph{transport properties}. The
theoretical description of such properties is 
known as transport theory.


%\subsection{Transport properties}
% Nevn noe av Enskog-informasjonen fra neste seksjon her.
% Generelt om transportegenskaper, enskogteori som en metode for å finne dem
% Kanskje nevne andre metoder for å regne ut transportegenskaper?
% Viskositet som transportegenskap
The transport properties of fluids are diffusion, 
thermal conductivity and viscosity. All of which
are associated to a gradient in certain physical
quantities in the system. Diffusion is a flux of 
mass due to a density gradient. Thermal conductivity 
is a flux of energy due to a temperature gradient.
Lastly, viscosity is a momentum flux due to a
velocity gradient. 


\subsection{Kinetic theory}
\label{sec:theory:kinetic_theory}
% TODO: Skriv en litt generell intro.
Kinetic theory describes fluids as a large 
collection of small particles. The properties 
of the fluid are then derived from the properties 
of its component particles. Kinetic theory allows 
calculation of the transport properties of fluids, 
given certain assumptions. This section will outline 
some relevant results and foundations of kinetic 
theory. Necessary assumptions and approximations 
will be a central topic, in order to give an 
understanding of when the theoretical descriptions 
are valid.
%Many quantities in thermodynamics
%and fluid dynamics can be computed using the
%framework of kinetic theory. The following 
%section outlines some key models and assumptions 
%in kinetic theory, focusing especially on Enskog 
%theory.

The Boltzmann equation is a fundamental equation
in kinetic theory. It describes conservation 
laws in six dimensional phase space (position 
and velocity). Its derivation is out of scope 
for this report. However, the assumptions made 
during the derivation are important. 

In the derivation of the Boltzmann equation, the 
fluid is assumed to be dilute. Furthermore, the 
particles interact in collisions that last for 
an infinitesimally short time. Lastly, particles 
are assumed to be uncorrelated between the collisions.

The latter assumption means that there is 
no correlation between different collisions. 
This is known as the assumption of molecular 
chaos. Note that if the fluid density is high, 
or the collision durations are not small, then 
the assumption of molecular chaos also breaks 
down. In other words, the above assumptions 
are interdependent.

Note that the Boltzmann equation will not be 
valid if the particles interact at long ranges, 
as they will be correlated between collisions.

%It means that the mean free time between 
%collisions is much larger than the collision 
%duration. 

%The assumption of molecular chaos breaks 
%down at high fluid densities, when collisions are 
%too frequent to be uncorrelated. 
%Furthermore, it does not work for real fluids 
%with long range continuous interaction potentials, 
%since long range interactions make the particles highly 
%correlated. 
%However, the assumption of molecular 
%chaos is more acceptable for low density HS-fluids. 

% Enskogteori:
Enskog theory is a description of fluids derived 
from kinetic theory. In Enskog theory, hydrodynamic
conservation laws can be derived from the Boltzmann
equation of kinetic theory. When deriving the Navier-
Stokes equations in Enskog theory, theoretical 
expressions for the transport coefficients can 
be obtained. These are only \emph{phenomenological 
constants} in the continuum mechanical derivation 
of the Navier- Stokes equations.

In Enskog theory, the fluid is assumed to consist 
of particles that only collide elastically. All 
the assumptions on which the Boltzmann equation
is based must also be satisfied.

% Antakelser
% Likevekt

%There are several microscopic properties which 
%define the properties of the macroscopic fluid.
%How the particles interact with each other is
%one important such property.




The remaining parts of this report will focus only on viscosity.
% Should I include macroscopic flux as a transport property?


\section{Hard-sphere fluids}
Hard-sphere (HS) fluids are fluids consisting of spherical particles 
interacting via the non-continuous potential 
\[
    \label{eq:hard_sphere_potential}
    u(r) = 
    \begin{cases}
        \infty, & r < \sigma \\
        0, & r > \sigma.
    \end{cases}
\]
$u$ is the interaction potential between two 
particles, $r$ is the distance between them, 
and $\sigma$ is the diameter of the particles. 
Upon hitting each other, hard spheres will 
undergo a perfectly elastic collision.

In the low-density regime, the fundamental 
assumptions made in Enskog theory 
\ref{sec:theory:kinetic_theory} are satisfied
for a HS fluid. Thus, Enskog theory provides a 
theoretical description of HS fluids. 

% This is not too consistent with the first section.
Because of this, the HS potential makes it 
possible to analytically describe certain fluid 
properties using statistical mechanics. The HS 
potential is especially useful when modelling 
transport properties. In the following sections, 
analytical expressions for the viscosity of HS 
fluids are presented.


\section{The viscosity of hard-sphere fluids}
One expression for the viscosity of HS fluids 
is the Enskog equation. This equation applies 
to one-component HS fluids consisting only of 
particles with the same mass $m$ and radius 
$\sigma$. The viscosity of such fluids is 
\cite{ref:pippo:composition_dependence}
\[
    \label{eq:enskog_viscosity}
    \eta(\rho,T) 
        = \eta_0 \left[g^{-1}(\sigma) 
        + 0.8\, V_{\text{excl}}\,\rho 
        + 0.776 \,V_{\text{excl}}^2 \,\rho^2 g(\sigma) 
        \right].
\]
\iffalse
\[
    \begin{split}
        \eta(n,T)
            &= \frac{\eta^0}{\chi} (1+\frac{1}{2} \alpha \chi n)^2 
            + \frac{3}{5} \bar{\omega}, \quad \text{where}\\
        \alpha
            &= \frac{8}{15} \pi \sigma^3, \quad \text{and} \\
        \bar{\omega}
            &= \frac{4}{9} n^2 \sigma^4 \chi \sqrt{\pi m \kB T}.
    \end{split}
\]
\fi
Here, 
\[
    \eta_0 = \eta(0, T) = 
    \frac{5}{16\sigma^2} \sqrt{\frac{m \kB T}{\pi}}
\]
is the viscosity of the fluid in the zero-density 
limit. $V_{\text{excl}}$, the excluded volume, 
is the part of the fluid which a particle cannot 
occupy, because it is occupied by other particles. 
$V_{\text{excl}}$ is commonly assumed to be the 
volume of all particles in the fluid. This however, 
is only correct in the zero-density limit. 
% When the fluid density is higher, 
The space in between particles may also be unavailable.
% TODO: Make a figure.

Lastly, in equation \eqref{eq:enskog_viscosity}, 
$g(\sigma)$ is a radial distribution function at 
contact.  $g(\sigma)$ is the probability distribution 
of particles around one particle in the fluid. 
This determines how the collision frequency depends 
on the density of the fluid. $g(\sigma)$ can be 
found for example using the system's equation of 
state, as done in \cite{ref:pippo:composition_dependence}.
This method is referred to as Modified Enskog Theory.

% Deler av dette kan også nevnes i seksjonen om transportegenskaper. 
% Det er et generelt problem det du beskriver om MFT (mean free time).


\section{Viscosity of hard-sphere fluid mixtures}
The Enskog equation (equation \eqref{eq:enskog_viscosity}) 
was generalized by Thorne to describe two-component fluid 
mixtures\cite{ref:chapman:non_uniform_gases}.
A further generalization to mixtures of arbitrary 
component numbers has been performed by Tham and 
Gubbins\cite{ref:tham:fluid_mixtures}. The results 
are outlined below, as presented in 
\cite{ref:pippo:composition_dependence}.

The viscosity of a dense binary mixture 
of two hard-sphere fluids is given by
\[
    \label{eq:thorne_viscosity}
    \eta_{\text{mix}} 
        = \left(
            \frac{y_1^2}{H_{11}} 
            + \frac{y_2^2}{H_{22}} 
            - \frac{2 y_1 y_2 H_{12}}{H_{11} H_{22}}
        \right)
        \left(
            1 - \frac{H_{12}^2}{H_{11} H_{22}}
        \right)^{-1}
        + \frac{3}{5} \bar{\omega}_{\text{mix}},
\]
where
\[
    y_1 
        = x_1 \left(
            1   + \frac{1}{2} x_1 \alpha_{11} \chi_{11} n 
                + \frac{m_2}{m_1 + m_2} x_2 \alpha_{12} \chi_{12} n
        \right), 
\]
and
\[
    \begin{split}
        H_{12} &= H_{21}
                =   -\frac{2 x_1 x_2 \chi_{12}}{\eta^0_{12}}
                    \cdot \frac{m_1 m_2}{(m_1 + m_2)^2}
                    \left( \frac{5}{3A^*_{12}} - 1 \right), \\
        H_{11}
                &=  -\frac{x_1^2 \chi_{11}}{\eta^0_1}
                    +\frac{2 x_1 x_2 \chi_{12}}{\eta^0_{12}}
                    \cdot \frac{m_1 m_2}{(m_1 + m_2)^2}
                    \left( \frac{5}{3A^*_{12}} + \frac{m_2}{m_1} \right). \\
    \end{split}
\]
$y_2$ and $H_{22}$ follow from exchanging the subscripts in $y_1$ and $H_{11}$ 
respectively.
$A^*_{12}$ is a dimensionless ratio of collision integrals (of type ${ij}$).
For hard spheres, $A^*_{12}$ is exactly unity, and for other forms of 
interaction, it is close to unity. % Source here is de Pippo. Should I point that out?

$\chi_{ij} = \chi_{ji}$ are radial distrubution 
functions for molecules of type $i$ colliding 
with molecules of type $j$.
They correspond to the one-component radial 
distribution function $g(\sigma)$ in equation 
\eqref{eq:enskog_viscosity}.
Finally, $\bar{\omega}_{\text{mix}}$ can be written
\[
    \begin{split}
        \bar{\omega}_{\text{mix}} 
            &= x_1^2 \bar{\omega}_{11} 
            + x_1 x_2 \bar{\omega}_{12} 
            + x_2^2 \bar{\omega}_{22}, \text{where}\\
        \bar{\omega}_{ij} 
            &= \frac{4}{9} n^2 \sigma_{ij}^4 \chi_{ij} 
            \sqrt{\frac{2\pi m_1 m_2 \kB T}{m_1 + m_2}} 
            \text{for} i,j = 1,2.
    \end{split}
\]

% TODO: Elaborate on this, after discussing with Roberto.
%       In particular, say something about the physical meaning of every
%       "Russian doll" in the equation.

% TODO: Should I include the multi-component equation as well?
%       Or maybe just include one and many components, and skip two-component?
%       ANSWER: Start with multi-component, then reduce to two.
%       COMMENT TO ANSWER: Since the two-component version is more important
%       to this project, and since it was derived first, I think it's more
%       natural to present the two-component equation first, like Pippo does.

It is instructive to demonstrate how the two-component 
Thorne equation \eqref{eq:thorne_viscosity} reduces to 
the one-component Enskog equation \eqref{eq:enskog_viscosity}.
This is done by setting the density of either of the 
components to zero. Setting \(x_2 = 0\), so that 
\(y_2, \omega_\text{mix} = x_1^2\bar{omega}_{11}\)
gives 
\[
    H_{ij} = 
    \begin{bmatrix}
        H_{11}  & 0 \\
        0       & 0
    \end{bmatrix}.
\]
Now, equation \eqref{eq:thorne_viscosity} reduces to 
\[
    \eta_{\text{mix}} 
        = \frac{y_1^2}{H_{11}} 
        + \frac{3}{5} \bar{\omega}_{\text{mix}}
        = \frac{\eta^0_1}{\chi_{11}} \left(
            1   + \frac{1}{2} x_1 \alpha_{11} \chi_{11} n 
        \right)^2
        + x_1^2\bar{\omega}_{11}
\]
Renaming the factors, this equals
\[
    \begin{split}
    \eta_{\text{one}} 
        &= \frac{\eta^0}{g(\sigma)} \left[
            1+\frac{1}{2} \alpha g(\sigma) \rho
        \right]^2
        + x^2\bar{\omega} \\
        %&= \frac{\eta^0}{g(\sigma)}\left[
        %    1+\frac{1}{2} \alpha g(\sigma) \rho 
        %    + \frac{1}{4} \alpha^2 g^2(\sigma) \rho^2
        %\right] \\
        &= \eta^0\left[
            g^{-1}(\sigma) 
            + \frac{1}{2} \alpha \rho 
            + \frac{1}{4} \alpha^2 g(\sigma)
        \right]
        + x^2\bar{\omega}, \\
    \end{split}
\]
Lastly, 
\[
    \begin{split}
    \alpha
        &= \frac{8}{15} \pi \sigma^3, \quad \text{and} \\
    \bar{\omega}
        &= \frac{4}{9} \rho^2 \sigma^4 g(\sigma) \sqrt{\pi m \kB T}
    \end{split}
\]
is inserted, and it is assumed that                 % TODO: Calculate this by hand!
\(V_{\text{excl}} = \frac{N \pi \sigma^3}{6}\) 
is the total volume of the particles in the fluid.
This gives equation \eqref{eq:enskog_viscosity}
\[
    \eta 
        = \eta_0 \left[g^{-1}(\sigma) 
        + 0.8   \, V_{\text{excl}}      \,\rho 
        + 0.776 \, V_{\text{excl}}^2    \,\rho^2 g(\sigma) 
        \right].
\]

